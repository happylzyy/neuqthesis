\section{绪论}
\subsection{引言}

镁及其合金是目前工业应用中密度最低的工程金属材料,密度在1.75~1.90g/cm3之间。镁合金比强度和比刚度高、具有优良电磁遮蔽性和的抗冲击性。并且易于加工,有良好的机械性能,制备简单、方便,可回收性高,因而被誉为“21世纪绿色结构材料”[1-2]。镁合金被用作各种优质工业合金的基础材料,在于航空航天、国防、电子、交通运输、尤其是汽车工业等领域有着广泛的应用前景。\par
改变光子晶体光纤中的几何参数,如空气孔的大小、位置、排列,纤芯的尺寸和位置,可以将光子晶体光纤的传输光谱、模式形状、非线性、色散和双折射等因素调整到传统光纤无法达到的值。或者在空气孔中选择性填充各式各样的材料,使得光波与所填充的材料发生相互作用,这种相互作用产生了新的传感应用。由于光子晶体光纤结构和填充材料的多样性,光纤传感领域涌现了大量的新研究和新应用。空气孔中填充金属,利用金属表面等离子体共振效应与光波发生相互作用,制成多种折射率光子晶体光纤传感器;调整空气孔大小,形成特殊的高双折射结构,利用sagnac干涉效应制成传感器;空气孔选择性填充液晶,利用模式谐振耦合制作传感器,另外,其他原理,如马赫曾德尔干涉、磁流体填充等多样的传感方式也得到了国内外研究机构的大量研究和应用。\par
液晶是一种由几何上呈各向异性的分子组成的有机材料,介于各向异性晶体和各向同性液体之间。液晶的种类很多,按照液晶产生的条件可以分为热致液晶和溶致液晶,热致液晶又包括向列相、近晶相和胆甾相。以向列相为例,向列相是最简单的液晶相,这种液晶也是应用最广泛的液晶材料。这类液晶通常由长条棒状分子组成,分子之间互相平行排列,具有很高的流动性,分子通常沿流动方向取向。外界电场和磁场对其取向具有很大的影响,外界磁场和电场发生变化时,液晶分子取向发生变化,液晶材料的介电常数、电导率和折射率等参量也随之变化,这就是液晶的电光效应。液晶在我们生活中有着广泛的应用,各类电子终端都会用到液晶显示屏,通过改变电压可以改变液晶的双折射性质或旋光状态,再经过彩色滤光片液晶屏可以显示图像。同样地,液晶也可以填充到光子晶体光纤中,制成多种用途的设备,如偏振分光镜、电调制光开关、偏振滤波器、传感器等等。\par

\subsection{镁及其合金特性及应用}

\subsubsection{镁及其合金的主要特点}
镁是一种比铝还轻的轻金属,密度约为1.74g/cm3,熔点约为650℃。镁的晶体结构为密排六方,是地壳中存储较多的金属之一,达到2.1\%,占据第三位,仅次于Al和Fe。镁广泛存在于地壳中,同时大量储存于海水中。\par
\subsubsection{镁合金的分类}
按工艺来分,镁合金有变形镁合金和铸造镁合金两大类,一般来说,通过锻造、轧制、挤压等方法制备的镁合金被称为变形镁合金,产品往往用于汽车行业、航空航天及军工行业。铸造镁合金的特点是可在相同的强度条件下减轻工件质量,可代替铝合金铸件在汽车零件和电器构件中的应用。变形镁合金在变形加工过程中通过成型工艺的选择和控制、结合热处理工艺,可使其显微组织细化,消除缺陷,可获得更高的强度、更好的延展性和更多样化的力学性能,从而满足更复杂多变工程结构件的应用需求,所以在实际生产中对变形镁合金的需求更大,对变形镁合金的研究也是当下研究热点。\par
\subsection{层状复合轧制工艺研究现状}
\subsubsection{复合轧制工艺概述}
复合轧制是将两层或多层金属或合金叠放在一起进行轧制,依靠原子之间金属键的相互吸引力而使金属复合的一种工艺。复合轧制可使两种或多种物理、化学和力学性能不同的材料在界面上实现牢固的冶金结合。目前以发展了多种复合工艺,包括热轧复合、冷轧复合、爆炸复合、异步轧制复合和堆焊等方法。\par
\subsection{铝/镁层状金属复合轧制研究现状}
目前国内外对铝/镁复合轧制工艺的研究还未取得较大进展,研究表明,利用累积叠轧(Accumulative roll bonding)可以较好的实现镁、铝金属的复合。ARB属于金属材料的严重塑性变形(SPD)(正文中括号全角状态下的括号)工艺的一种,在循环中包括,多次轧制、切割、堆垛和固态变形粘合。使金属发生严重的塑性变形,产生大块超细的晶界结构。铝/镁层状金属利用累积叠轧技术制备,可有效提升复合板材的晶界结构,提高硬度等综合性能。它能够实现低成本效益高的连续制造,所以该技术具有很好的商业化潜力。\par
\subsection{本文的研究意义和主要内容}
因为镁合金具备许多优点,在诸多领域有着广泛的应用,然而应用的范围却被腐蚀性差等缺点限制。另一方面,纯铝具有良好的耐腐蚀性。通过轧制方法制成Al/Mg/Al复合板材,实现外层铝对中心层镁层的包覆。让复合材料兼具镁合金的高比强度和铝合金优良的耐腐蚀性能。同时,镁、铝的结合还能提升材料的整体塑性和变形抗力,延缓镁合金变形过程中裂纹的扩展。这种复合材料在航天航空以及汽车领域具有巨大的潜力。本文对于提升镁铝复合板材的生产水平、突破镁合金的应用局限有一定的现实意义。\par

\clearpage